%% ANEXOS
\hypertarget{estilo:anexo}{}

\chapter{ANEXO B - MANUAL DO MODELO EtaWS}
\label{anexoB}

Este anexo contém um documento cujo objetivo é apresentar uma descrição detalhada sobre o funcionamento e a utilização do modelo Eta Workstation (EtaWS) acoplado com o sistema de assimilação \textit{Rgeional Physical-space Statistical Analysis System} (RPSAS), que em conjunto pode ser chamado de EtaWS+RPSAS. Tem-se por intenção que a utilização deste documento sirva de referência para futuros utilizadores do sistema, seja como fonte de referência bibliográfica ou como fonte de informação sobre sua utilização. 

Neste anexo, são apresentadas também definições técnicas sobre o ambiente de execução no qual o modelo foi utilizado para a realização dos experimentos desenhados para esta dissertação de mestrado.

Este anexo está dividido nas seguintes subseções:

\begin{enumerate}
\item Introdução;
\item Ambiente de Execução;
\item Definições da Grade Horizontal;
\item Definição dos Níveis Verticais;
\item Definição da Orografia;
\item Pré-Processamento e Inicialização;
\item Pós-Processamento;
\item Outras Definições.
\end{enumerate}

\section{Introdução}
\label{ss:introeta}

Este documento de referência do sistema EtaWS+RPSAS foi adicionado como anexo desta dissertação de mestrado como uma contribuição aos futuros utilizadores da versão do modelo EtaWS e do sistema RPSAS, utilizados neste trabalho. Este manual foi adaptado do manual escrito por Chris Peters em 1996, originalmente da versão antiga do modelo Eta e do manual escrito por Matthew E. Pyle em 2001, sobre uma versão não hidrostática do modelo EtaWS, mais recente e portanto muito próxima à versão utilizada nos experimentos desta dissertação de mestrado (modelo EtaWS+RPSAS - modelo EtaWS acoplado ao sistema de assimilação RPSAS). Foi utilizado também o manual do modelo Eta utilizado pelo Laboratório Nacional de Computação Científica (LNCC), datado de 2004 e alguns documentos técnicos do CPTEC.

Este documento tem por objetivo fornecer informações específicas e mais detalhadas sobre o funcionamento e utilização do modelo Eta Workstation (EtaWS) sobre o domínio da América do Sul (AS). Este documento de referência poderá ser utilizado como referência para a futura implementação operacional do modelo EtaWS+RPSAS.

\section{Ambiente de Execução}
\label{ss:ambexeceta}

No parque computacional do Centro de Previsão de Tempo e Estudos Climáticos (CPTEC), o sistema EtaWS+RPSAS foi executado no cluste UNA, com 1100 processadores. O sistema operacional utilizado é o Linux comum com otimizado para clusters. No ambiente são utilizadas também rotinas especiais para paralelização do ambiente.

Nas subseções a seguir, são apresentadas os detelhes e definições do sistema EtaWS+RPSAS.

\section{Grade Horizontal}
\label{ss:gradehorizontal}

Uma das primeiras tarefas a serem realizadas com o modelo EtaWS, é a definição dos parâmetros da grade horizontal do modelo. Estes parâmetros são:

\begin{enumerate}
\item Localização;
\item Tamanho;
\item Resolução;
\end{enumerate}

O modelo EtaWS utiliza a grade semi-alternada E de Arakawa, assim como na versão antiga do modelo. Uma das principais características deste tipo de grade, é que os pontos de massa são alternados com os pontos de ventos da seguinte maneira, como ilustrado na figura a seguir:

FIGURA

A grade do modelo é configurada sobre uma grade de latitudes/longitudes rotacionadas, ou seja, as coordenadas terrestres são rotacionadas de forma que o centro da grade permaneça no ponto onde a linha imaginária do Equador cruza o primeiro meridiano. Para se definir a localização da grade E de Arakawa na terra, deve-se especificar as latitudes/longitudes geodésicas do ponto central da grade.

Para se definir os parâmetros de localização tamanho e resolução da grade do modelo, os seguintes parâmetros devem ser explícitamente especificados nos arquivos de parâmetros, que são utilizados para a compilação do modelo:

\begin{itemize}
\item $IM$: Número de pontos de massa ao longo da primrira linha da grade;
\item $JM$: Número de linhas norte-sul (=5 no exemplo);
\item $IMJM$: Número total de pontos de massa em ponto de grade = $IM*JM-JM/2=18$ (no exemplo acima);
\item $TLM0D$: Longitude geodésica (+=deg E) do centro da grade E;
\item $TPH0D$: Latitude geodésica (+=deg N) do centro da grade N;
\item $DLMD$: Incremento oeste-leste na grade transformada (em graus);
\item $DPHD$: Incremento norte-sul na grade transformada (em graus);
\item $WBD$: Longitude transformada da fronteira oeste da grade do modelo;
\item $SBD$: Latitude transformada da fronteira sul da grade do modelo.
\end{itemize}

Para a região do Brasil, pode-se ajustar o modelo com uma resolução de 20 km com os seguintes parâmetros:

\begin{table}[htbp]
\caption{Exemplos de configurações dos parâmetros de grade para \-di\-fe\-ren\-tes resoluções da grade do modelo EtaWS.}
\begin{tabular}{l|l|r|l|r|r|r}
\hline
 & 80 km & \multicolumn{1}{l|}{48 km} & 40 km & \multicolumn{1}{l|}{32 km} & \multicolumn{1}{l|}{29 km} & \multicolumn{1}{l}{15 km} \\ \hline
DLMD (deg) & 15/26 & 1/3 & \multicolumn{1}{r|}{5/18} & 7/1 & 7/1 & 3/31 \\ \hline
DPHD (deg) & 14/26 & 4/13 & 15/57 & 1/5 & 5/27 & 2/21 \\ \hline
\end{tabular}
\label{tabB01}
\end{table}

$WBD$ e $SBD$ são funções do tamanho e resolução da grade. O programa WSB.all calcula estes parâmetros.

Em alguns códigos o parâmtro TL0MD é definido como graus nagativos de longitude oeste e, em outros códigos como graus positivos de longitude oeste e em outros é definido como graus positivos de longitude leste.

No código do modelo EtaWS, a variável IMJM é utilizada para indexação. IMJM é incrementada linha por linha para cada par de pontos de massa-vento (H-V). No exemplo acima, seja K=1,IMJM. No código do modelo, algumas variáveis são indexadas por (IMJM,LM), onde LM é o número de níveis verticais do modelo. T(1,1) é o valor da temperatura no meio do nível mais alto do modelo no ponto mais a sudeste H. u(1,1) é a componente $u$ do vento no topo da camada média no ponto V à direita do ponto K=1 H. Observe que enquanto o ponto K=4 H está no canto sudeste da grade, o ponto k=4 V está localizado na próxima linha do limite lateral oeste.

\subsection{Coordenada Vertical}
\label{ss:coordenadavertical}

A coordenada vertical do modelo EtaWS é uma generalização da coordenada sigma $\sigma$. Embora ambas as coordenadas, eta ($\eta$) e sigma ($\sigma$) são baseadas na pressão e normalizadas de 0 a 1, a coordenada eta é normalizada em relação à pressão do nível do mar, e a sigma em relação à superfície de pressão. As superfícies resultantes são \textit{quasi}-horizontais.

\subsubsection{Distribuição vertical das camadas}

O ajuste do número de camadas vertical do modelo, fica a critério do usuário. O NCEP, por exemplo, já utilizou algumas versões do modelo Eta com 16 a 120 camadas, geralmente aumentando-se também a resolução horizontal e a capacidade do computador. O usuário precisa gerar os seguintes arquivos, os quais serão utilizados como arquivos "INCLUDE" para o código FORTRAN do modelo ou como arquivos de namelists, que são lidos pelo código do modelo:


