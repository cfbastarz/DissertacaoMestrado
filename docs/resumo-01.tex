%% RESUMO
\begin{resumo}
\hypertarget{estilo:abstract}{}

Neste trabalho de dissertação de mestrado são apresentados os resultados da implementação da Assimilação de Precipitação do TRMM no modelo de previsão EtaWS. O esquema de assimilação de precipitação foi implementado com o objetivo de avaliar o impacto produzido pela assimilação de precipitação do TRMM na análise gerada pelo sistema RPSAS e nas previsões produzidas pelo modelo EtaWS, através de uma estudo de caso da simulação de CCMs. O uso da assimilação de precipitação do TRMM pelo modelo EtaWS resulta em um incremento substancial na previsão de curto prazo (tipicamente de 6 horas) e consequentemente na análise produzida pelo sistema de assimilação RPSAS. O desempenho do modelo EtaWS com o esquema de assimilação de precipitação na previsão de 24 horas, é melhorado apresentando valores de Viés e Erro Quadrático Médio menores, se comparado com o mesmo modelo sem a assimilação de precipitação. Foram verificados os campos de vento (zonal e meridional), altura geopotencial, umidade relativa e temperatura do ar. Para estes campos observa-se que, em geral, o modelo EtaWS com a assimilação de precipitação é capaz de simular um caso de CCM ocorrido em 23 de Janeiro de 2003. Neste estudo, foi comparada a precipitação produzida pelo modelo EtaWS com a precipitação observada do SALLJEX. Foram comparados também os prefis de vento simulado pelo modelo e os da Reanálise 2 do NCEP/DOE durante a ocorrência do CCM e encontrou-se uma maior compatibilidade entre os perfis. Os fluxos meridionais de umidade também se mostram mais compatíveis, tendo a mesma direção do vento e a intensidade do fluxo valores dentro da mesma ordem de magnitude. Os resultados desta simulação mostram também que, além do modelo ser capaz de reproduzir o padrão de precipitação observado do CCM, os níveis pluviométricos produzidos pelo modelo de previsão, também são compatíveis com o observado. A partir destes resultados torna-se real a possibilidade de se implementar operacionalmente a assimilação de precipitação do TRMM no CPTEC e contribuir com a acurácia da previsão de curto prazo produzida pelo centro.

\end{resumo}
