%% RESUMO
\begin{resumo}
\hypertarget{estilo:abstract}{}

O presente trabalho apresenta os resultados de um estudo de impacto da assimilação de dados de precipitação estimada do \textit{Tropical Rainfall Measuring Mission} (TRMM) no \textit{Regional Physical-space Statistical Analysis System} (RPSAS) do Centro de Previsão de Tempo e Estudos Climáticos (CPTEC), durante  o mês de janeiro de 2003. Neste estudo, a assimilação de dados de precipitação foi realizada durante a geração do \textit{first guess} pelo modelo regional Eta de 20 km do CPTEC utilizando-se uma metodologia similar ao \textit{nudging}. A partir deste \textit{firt guess}, o sistema de assimilação de dados RPSAS gera uma análise com resolução horizontal de 20 km sobre um domínio que cobre grande parte da América do Sul, a partir da qual foram produzidas previsões de 24 horas utilizando o modelo Eta. Dados da reanálise global 2 do NCEP/DOE e regional de 40 km do CPTEC foram utilizados para comparação com as análises geradas com e sem a assimilação de dados de precipitação. Dados coletados durante a campanha do \textit{South America Low-Level Jet EXperiment} (SALLJEX), do \textit{Global Precipitation Climatology Project} (GPCP) e TRMM foram utilizados para comparação com as previsões do modelo Eta. A avaliação dos resultados foi feita com base em índices estatísticos e mostrou que o desempenho das previsões de até 24 horas do modelo Eta, principalmente nas primeiras horas, é sensivelmente melhor com a inclusão da precipitação, apresentando valores de Viés e Erro Quadrático Médio menores, se comparado com o mesmo modelo sem a assimilação de precipitação. Foi realizado também um estudo de caso de Complexo Convectivo de Mesoescala (CCM) ocorrido durante a campanha do (SALLJEX). A inclusão da assimilação de precipitação no ciclo do sistema Eta+RPSAS permitiu simular um caso de CCM ocorrido em 23 de Janeiro de 2003 com maiores detalhes, onde foi comparada a precipitação produzida pelo modelo Eta com a precipitação observada durante o SALLJEX e também com o GPCP e TRMM. Foram comparados também os perfis de vento das análises do RPSAS com a reanálise do NCEP durante a ocorrência do CCM e encontrou-se maior concordância entre os perfis, em relação aos perfis produzidos sem a assimilação de precipitação. Os fluxos meridionais de umidade calculados também mostraram valores semelhantes aos valores produzidos pela reanálise do NCEP. Os resultados destas simulações mostram também que além do modelo Eta, com assimilação de precipitação, ser capaz de reproduzir com maior detalhamento algumas das principais características do CCM, como a distribuição espacial de precipitação observada, os níveis pluviométricos produzidos pelo modelo de previsão foram melhorados. Neste estudo foi possível também verificar que o uso da assimilação de precipitação do TRMM na geração do \textit{first guess} melhorou a análise regional produzida pelo sistema de assimilação RPSAS do CPTEC e, consequentemente, incrementou a previsão de 24 horas produzida pelo modelo Eta do centro. A partir dos resultados obtidos e, realizando-se alguns ajustes ao esquema de assimilação propostos, pode-se estudar a possibilidade de se implementar operacionalmente a assimilação de precipitação do TRMM no sistema de assimilação/previsão regional Eta+RPSAS do CPTEC com o objetivo de contribuir para o aumento do realismo das previsões regionais do centro, principalmente as de curto prazo além de incrementar esse sistema com o aproveitamento de mais uma fonte de dados de observações de precipitação.

\end{resumo}