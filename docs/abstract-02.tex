%% ABSTRACT
\begin{abstract}
\hypertarget{estilo:abstract}{}

In this work a data assimilation study was performed to assess the impact of TRMM's estimated precipitation assimilation on the CPTEC's RPSAS analyses and the Eta model forecast over the east Andes, during a case of MCC ocurred between 22th and 23th January 2003. The RPSAS data assimilation system and the Eta mesoscale regional model (both with 20 km of horizontal resolution) were run together with and without the TRMM precipitation. In this study, the precipitation assimilation is basically a nudging approach and is performed during the first guess  by the Eta model, like in the NCEP EDAS precipitation data assimilation. During this process, the model adjusts the precipitation by comparing, at which grid point an at which time step, the model precipitation against the TRMM precipitation. Doing this, some adjustments are made on the latent heat vertical profile, water vapor mixing ratio and relative humidity, and the precipitation is calculated again by the Betts-Miller-Janjić convective parametrization. On the next step, the RPSAS produces an analysis which covers most of the South America and the adjacent oceans. From this analysis the Eta model produces 6h, 12h, 18h and 24h forecasts. Data collected from the SALLJEX was used to compare the model forecasts and the CPTEC 40 km Regional Reanalysis was used to compare the model forecasts with the RPSAS analyses. Also were compared the wind profiles simulated by the Eta model and the NCEP/DOE Reanalysis 2 during the occurrence of the MCC and has found a better fit between these profiles. The moisture flux at the east Andes showed similar values to those observed. The results of this simulation also show that, beyond the model is capable of reproducing the observed pattern of the MCC precipitation, the rainfall levels produced by the model, were consistent with the observed. In this study were noticied that the use of the TRMM precipitation assimilation improved the first guess regional analysis produced by the RPSAS assimilation system. These results show that the TRMM estimated precipitation assimilation contributes to short-term predictions produced by the Eta model and has a positive impact on the analysis generated by the CPTEC's RPSAS system. With this, it becomes a real possibility of its operational implementation contributing to the increased accuracy of regional weather forecast produced by the center.

\end{abstract}