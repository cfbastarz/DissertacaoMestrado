%% AGRADECIMENTOS
\begin{agradecimentos}
\hypertarget{estilo:agradecimentos}{}

Em primeiro lugar, agradeço a Deus pela vida. Aos sentimentos que me permitem amar e fazer feliz. Às pessoas que assim o fazem sem a pretensão de seguir-me.

A meus pais, Friedrich e Mirian, que me criaram e me proveram com seu amor, carinho. À minha irmã Beatriz pela sua cumplicidade, respeito e amor.

À Helena, pelo seu amor, paciência, cumplicidade e inefável companheirismo.

Aos amigos Weber, Rogério, Isabel Pilotto, Diego, Marília, Ricardo, Paulo, Dayana, Jairo, Francisco e Tania. Agradeço, em primeiro lugar, pela amizade conquistada. Agradeço também, imensamente, à ajuda que me prestaram nos momentos em que mais precisei. Obrigado pelo companheirismo.

À Rita e Tatiane, pela ajuda nos trabalhos e pelos exemplos de vida.

Aos meus professores de graduação Aury, Maria de Fátima, Vera Lia, Angela, Geraldo e Tania. Agradeço pela confiança e orientação, pelas recomendações e oportunidades que me deram.

Aos meus orientadores de mestrado Dirceu e Pablo, pela orientação e ajuda. Obrigado pela cumplicidade, pela confiança que creditaram em mim e pelos ensinamentos.

Ao João, pela amizade, pela ajuda prestada com os diversos programas, scripts, modelos, ideias e conceitos. Ao Sapucci pelo grande auxílio na manutenção do RPSAS, gerenciamentos dos scripts, ideias e observações.

Ao grupo de Assimilação de Dados do CPTEC/INPE, pelo apoio durante as reuniões do GEDAI, pela iniciativa e comprometimento.

Ao Marcos Mendonça e à Renata pela ajuda com o programa para acumular a precipitação do TRMM.

Ao Rildo pelo seu auxílio com os scripts e explicações sobre a avaliação dos modelos de previsão.

Às secretárias do Curso de Pós-Graduação em Meteorologia do INPE, Lilian, Fabiana e Simone. Ao Gilson, Marcela e Michele, secretários da DMD. Obrigado pelos auxílios que prestaram nos momentos burocráticos, pela atenção e simpatia.

À PGMET pela ajuda concedida a mim em momentos difíceis durante a minha caminhada.

A todos os colegas que, embora não citados aqui, colaboraram para o desenvolvimento deste trabalho, seja com sugestões e apontamentos ou simplesmente pela compreensão de quem sou.

Ao CPTEC/INPE pelo uso de suas instalações para o desenvolvimento deste trabalho.

À Comunidade do Software Livre e às pessoas que dispõem de seu tempo livre para construir os programas e scripts, sem os quais, boa parte da pesquisa realizada em diversas áreas seria mais complicada e demorada. 

À FAPESP (Fundação de Amparo à Pesquisa do Estado de São Paulo) pela concessão da bolsa de mestrado e financiamento do projeto No. 2007/06256-6.

\end{agradecimentos}
