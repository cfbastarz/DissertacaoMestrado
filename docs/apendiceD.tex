%% ANEXOS
\hypertarget{estilo:anexo}{}

\chapter{ANEXO D - ÍNDICES ESTATÍSTICOS}
\label{anexoD}

Para o cálculo o Viés (\autoref{form10}), calculou-se as diferenças entre os campos de cada experimento considerado em relação à Reanálise 2 do NCEP/DOE. A seguir foi calculada a média do campo para o domínio de 50.2$ºS$ 10$ºN$ e 83$ºW$ 25.8$ºW$, respeitando-se cada saída do modelo 0, 6, 12, 18 e 24h. 0h Indica o valor da análise e 6h o valor do \textit{first guess}. O cálculo do EQM (\autoref{form11}) foi feito com base nos valores calculados do Viés. Para isso, tomou-se o quadrado as diferenças entre as previsões dos experimentos e a Reanálise. Para o cálculo da CA (\autoref{form12}), foram calculadas as anomalias dos experimentos (SAP e CAP) e a anomalia da Reanálise.

\begin{equation}
Vies=E_{i}-R_{i}
\label{form10}
\end{equation}

\begin{equation}
EQM=(E_{i}-R_{i})^{2}
\label{form11}
\end{equation}

\begin{equation}
CA=\frac{\sum_{i=1}^{5}\{[(E_{i}-C)-\overline{(E_{i}-C)}]*[(R_{i}-C)-\overline{(R_{i}-C)}]\}}{\sqrt{\sum_{i=1}^{5}[(E_{i}-C)-\overline{(E_{i}-C)}]^{2}*\sum_{i=1}^{5}[(R_{i}-C)-\overline{(R_{i}-C)}]^{2}}}*100
\label{form12}
\end{equation}

Onde:

\begin{itemize}
\item $E$: representa um dos experimentos (SAP ou CAP);
\item $R$: representa a Reanálise 2 do NCEP/DOE;
\item $C$: representa a Climatologia;
\item $E-C$: representa a anomalia dos experimentos (SAP ou CAP);
\item $\overline{E-C}$: representa a média da anomalia dos experimentos (calculada com base no número de dias de avaliação);
\item $R-C$: representa a anomalia da Reanálise;
\item $\overline{R-C}$: representa a média da anomalia da Reanálise (calculada com base no número de dias de avaliação).
\end{itemize}

O índice $i$, varia de 1 a 5 e indica o tempo de previsão 0, 6, 12, 18 e 24h horas. No caso da reanálise, este índice indica a análise correspondente à hora de previsão do experimento. Para a climatologia, este índice não varia, visto que se utiliza um valor fixo para a hora sinótica avaliada (00Z e 12Z).
