%% ANEXOS
\hypertarget{estilo:anexo}{}

\chapter{ANEXO C - DESEMPENHO COMPUTACIONAL E \textit{SCRIPTS}}
\label{anexoC}

O sistema Eta+RPSAS utilizado neste trabalho de dissertação foi executado em um \textit{cluster} (de \textit{hostname} UNA) do pólo computacional do CPTEC. Esta máquina possui um total de 1100 processadores, dos quais 100 são reservados à operação e os outros 1000 restantes à pesquisa, com 250 nós à disposição (4 processadores por nó).

Os \textit{jobs} do sistema Eta+RPSAS (\textit{scripts} de execução submetidos à máquina) são otimizados para processamento paralelo e aguardam em fila para o encaminhamento em cada nó, quando são necessários 4 ou mais processadores. Quando o \textit{job} requer menos do que 4 processadores, ele é submetido diretamente ao \textit{cluster} evitando a espera em fila.

Embora a paralelização possibilite o uso massivo de processamento e otimize o tempo de execução dos modelos, grande parte do custo computacional demandado pelo sistema Eta+RPSAS, é devido ao sistema de assimilação RPSAS, visto que o modelo Eta é extremamente rápido em suas operações. Muito embora o sistema RPSAS seja otimizado (do ponto de vista estatístico) se comparado com o sistema 3DVAR, o tempo de processamento reservado ao sistema se deve ao fato de que as operações do RPSAS são realizadas em cada ponto de grade. Além disso, a concorrência pelo uso do \textit{cluster} no CPTEC é intensa o que acaba por aumentar o tempo de espera em fila em determinadas situações.

Em termos de armazenamento, foram utilizados em torno de 1 TB para o armazenamento das condições iniciais do NCEP, condições de contorno do modelo global do CPTEC, dados de TSM, dados assimilados pelo RPSAS (no formato ODS), dados de precipitação do TRMM (no formato binário e em E-GRD), as análises geradas pelo RPSAS e o \textit{first guess} e as previsões do modelo Eta para as simulações de Janeiro de 2003. A \autoref{tab07}, resume a quantidade de espaço em disco utilizado para o armazenamento de 1 experimento.

\break

\begin{longtable}{c>{\centering}m{2.3cm}|>{\centering}m{1.5cm}|>{\centering}m{1.5cm}|>{\centering}m{2.2cm}|>{\centering}m{1.5cm}|>{\centering}m{2cm}}
\caption{Custo Computacional para 1 experimento.}
\label{tab07}
\endfirsthead
\cline{2-7} 
 & Job & Procs/

Nós & Tempo & Resultado & Espaço & Forçantes\tabularnewline
\cline{2-7} 
 & run1\_repsas & 8 procs/

2 nós & 30 min & Análises/

Obs em HDF/ODS & 22 GB & C. C. 100 GB\tabularnewline
\cline{2-7} 
 & run2\_gespre & 1 proc & 35 seg & --- & --- & C. I. 100 GB\tabularnewline
\cline{2-7} 
 & run3\_grdeta & 4 procs/

1 nó & 4 seg & --- & --- & TSM 20 MB\tabularnewline
\cline{2-7} 
 & run4\_mkbond & 1 proc & 1 seg & --- & --- & ODS 40 MB\tabularnewline
\cline{2-7} 
 & run5\_bctend & 1 proc & 40 seg & --- & --- & TRMM 200 MB\tabularnewline
\cline{2-7} 
 & run6\_fceta & 16 procs/

4 nós & 15 min & Previsões em GRB & 7,3 GB & ---\tabularnewline
\cline{2-7} 
 & run7\_poseta & 1 proc & 1 min & --- & --- & ---\tabularnewline
\cline{2-7} 
 & run8\_fgeta & 1 proc & 1 min & First Guess em HDF/BIN & 9,3 GB & ---\tabularnewline
\hline 
\multicolumn{1}{c|}{Total} & 8 jobs & --- & 48'09'' & --- & \multicolumn{2}{>{\centering}m{2cm}}{264,0 GB}\tabularnewline
\hline
\end{longtable}

Os valores mostrados na \autoref{tab07} são referentes a um experimento com duração de 1 mês (no caso Janeiro de 2003, considerando-se de 02 a 15 de Janeiro como período de \textit{spin up}), com previsões de 24 horas e saídas a cada 06 horas, sendo que nos horários das 00Z e 12Z são calculadas as previsões livres (24 horas) e às 06Z e 18Z as previsões de curto prazo (06 horas). Para uma estimativa destes valores para vários experimentos (para 1 mês), é necessário considerar-se que as forçantes serão mantidas constantes, i. e., seu tamanho será sempre o mesmo entre os experimentos.

\break

\textbf{SEQUENCIA DE EXECUÇÃO DOS PROCESSOS DO CAD (\textit{SCRIPTS})}

No sistema Eta+RPSAS, como esquematizado na \autoref{fig08}, adaptado de \citeonline{rogersetal96} e \citeonline{peters96}, são executadas as seguintes etapas:

\begin{enumerate}[1.]
\item \textbf{GESPRE:} nesta etapa as previsões do modelo global do CPTEC (condição de contorno do modelo Eta) são transformados de coeficientes espectrais (espaço espectral) para ponto de grade (espaço físico) e a coordenada vertical sigma do modelo global é interpolada na coordenada vertical do modelo Eta;
\item \textbf{GRDETA:} interpola o \textit{first guess} (LATLON) na grade do modelo Eta (E-GRID) usada como condição inicial. Interpola os campos de superfície (temperatura da superfície do mar, cobertura de neve e gelo) na grade do modelo Eta;
\item \textbf{MKBOND:} lê as previsões do modelo global em coeficientes espectrais em coordenadas sigma e as transforma em uma grade intermediária LAT/LON (latitude/longitude) em coordenadas eta, finalmente extrai valores laterais de contorno para a previsão do modelo Eta;
\item \textbf{BCTEND:} calcula as tendências laterais de contorno para previsão do modelo Eta;
\item \textbf{FCTETA:} executa o modelo Eta propriamente dito para calcular as previsões utilizando o FD (previsão livre de 24, 48 ou 72 h). Durante a geração do \textit{first guess} (previsão de 6 h) é feita a assimilação de precipitação pelo modelo Eta;
\item \textbf{POSETA:} realiza o pós-processamento do modelo Eta (transforma as previsões para o formato GRB e E-GRID);
\item \textbf{FGETA:}  submete os dados pós-processados E-GRID para o RPSAS;
\item \textbf{REPSAS:} executa o sistema RPSAS para gerar uma nova análise que servirá de condição inicial para o modelo Eta gerar uma nova previsão.
\end{enumerate}

As etapas de 1 a 4 são conhecidas por etapa de Pré-Processamento (em que são preparadas as condições iniciais e de contorno para a integração do modelo). A etapa 5 é conhecida como execução das previsões (a integração do modelo propriamente dita). A etapa 6 é a etapa de Pós-Processamento (em que as saídas das integrações são preparadas no formato final para manipulação e visualização). Na etapa 7, os dados do modelo Eta são transformados para o formato do RPSAS (normalmente em \textit{Hierarchical Data Format} - HDF). Finalmente, a etapa 8 realiza a combinação dos resultados da integração do modelo (\textit{first guess}) e observações (convencionais e não convencionais disponíveis naquele momento) que são utilizadas para gerar a análise. Operacionalmente no CPTEC, nas etapas de 1 a 6 são utilizados scripts semelhantes para as previsões regionais tanto de tempo como de clima \cite{fernandez10}.

O sistema Eta+RPSAS é executado de forma cíclica e contínua durante um intervalo de tempo determinado indicado ao sistema pelo usuário. O ciclo de assimilação/previsão do sistema Eta+RPSAS é mostrado com mais detalhes no diagrama a seguir. Este diagrama organiza os 8 principais processos desse ciclo que são disparados no cluster. Os nomes dos processos estão destacados em negrito. O número de processadores e o tempo médio de execução de cada etapa são mostrados entre parênteses além de uma breve explicação da função de cada processo.

\break

\begin{sideways}
\setlength{\unitlength}{1mm}
\begin{picture}(0,0)
\put(-120,-5){\textbf{Eta+RPSAS 20 CC T126L28}}
\put(-120,-15){Gera a análise para o Eta fazer}
\put(-115,-20){a previsão do próximo horário}
\put(-105,-28){\textbf{run1$\_$repsas.ksh}}
\put(-110,-33){(8 processadores, $\sim$ 15')}
\put(-91,-41){\textbf{1}}
\put(-90,-40){\circle{8}}

\put(-60,-37){Interpola a condição de contorno}
\put(-58,-42){nas grades horizontal e vertical.}
\put(-57,-50){\textbf{run2$\_$gespre.ksh}}
\put(-60,-55){(1 processador, $\sim$ 35")}
\put(-66.3,-51){\textbf{2}}
\put(-65,-50){\circle{8}}

\put(-108,-123){Calcula as tendências}
\put(-108,-128){laterais e de contorno}
\put(-105,-110){\textbf{run5$\_$bctend.ksh}}
\put(-109,-115){(1 processador, $\sim$ 40")}
\put(-91,-101){\textbf{5}}
\put(-90,-100){\circle{8}}

\put(-161,-79.5){Pós-processamento}
\put(-160,-66.5){\textbf{run7$\_$poseta.ksh}}
\put(-163,-71.5){(1 processador, $\sim$ 1')}
\put(-121,-71){\textbf{7}}
\put(-120,-70){\circle{8}}

\put(-52.3,-75){Interpola os campos de sup.}
\put(-52.3,-80){superfície na grade do modelo}
\put(-52.3,-63.5){\textbf{run3$\_$grdeta.ksh}}
\put(-55.3,-68.5){(4 processadores, $\sim$ 4")}
\put(-61,-71){\textbf{3}}
\put(-60,-70){\circle{8}}

\put(-56,-103.5){Extrai valores laterais}
\put(-48,-108.5){de contorno}
\put(-57.2,-90.5){\textbf{run4$\_$mkbond.ksh}}
\put(-57.9,-95.5){(1 processador, $\sim$ 1')}
\put(-66.5,-91){\textbf{4}} 
\put(-65,-90){\circle{8}}

\put(-157,-37){Prepara o FG para o}
\put(-162,-42){RPSAS gerar a análise}
\put(-153,-50){\textbf{run8$\_$fgeta.ksh}}
\put(-158,-55){(1 processador, $\sim$ 1')}
\put(-116.2,-51.4){\textbf{8}}
\put(-115,-50){\circle{8}}

\put(-154,-103.5){Calcula a previsão}
\put(-184,-108.5){Executa o Eta, com FD e faz a}
\put(-171,-113.5){Assimilação de Precipitação}
\put(-152,-90.5){\textbf{run6$\_$fceta.ksh}}
\put(-165,-95.5){(16 processadores, $\sim$ 15')}
\put(-116,-91.3){\textbf{6}}
\put(-115,-90){\circle{8}}
\end{picture}
\end{sideways}
