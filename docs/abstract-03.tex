%% ABSTRACT
\begin{abstract}
\hypertarget{estilo:abstract}{}

This work presents the results of a impact study of estimated precipitation data assimilation in RPSAS from CPTEC, during January 2003. In this study, the precipitation assimilation was performed during the generation of first guess by the regional Eta model using a methodology similar to nudging. From this first guess, the RPSAS system generates an analysis with 20 km horizontal resolution in an area that covers South America, from which were produced 24 hours forecasts using the Eta model. Global reanalysis data from NCEP/DOE and regional reanalysis data from CPTEC were used for comparison with the tests generated with and without the precipitation assimilation. Data collected during the SALLJEX campaign, TRMM and GPCP were used for comparison with the predictions of the Eta model. The evaluation of the results was made based on statistical indices and showed that the performance of forecasts up to 24 hours of the Eta model, especially in the early hours, is significantly enhanced by the inclusion of precipitation. The values of Bias and Mean-Square Error compared with the same model without the assimilation of precipitation were minor. Also was carried out a case study of Mesoscale Convective Complex occurred during the SALLJEX campaign. The inclusion of the precipitation assimilation in the Eta+RPSAS forecasts/analysis cycle allowed to simulate a case of MCC occurred on 23 January 2003 with more details. Was also compared the wind profiles from the RPSAS analysis with the NCEP reanalysis for the occurrence of the MCC and were found a greater concordance between the profiles in relation to the profiles produced without the precipitation assimilation. Moisture flux was also calculated and showed similar values to the values produced by the NCEP reanalysis. The results of these simulations also show that besides the Eta model with precipitation assimilation, is able to play in more detail some key features of the MCC, as the spatial distribution of observed precipitation. Also the rainfall levels produced by the forecast model have been improved. This experiment also found that the use of the assimilation of TRMM precipitation in the generation of first guess improved the regional analysis produced by the CPTEC's RPSAS assimilation system and therefore increased the 24 hours forecasts produced by the Eta model. From the results, and making a few adjustments to the proposed assimilation scheme, we can study the possibility to implement operationally the assimilation of TRMM precipitation in the CPTEC's Eta+RPSAS system in order to contribute to increasing the realism of the regional forecasts of the center, especially the short-term as well as increase the system with the use of one more source of precipitation observations.

\end{abstract}