%% RESUMO
\begin{resumo}
\hypertarget{estilo:abstract}{}

O presente trabalho apresenta os resultado de um estudo de impacto da assimilação de dados de precipitação do TRMM pelo sistema de assimilação regional de dados RPSAS do CPTEC em conjunto com o modelo de mesoescala Eta de 20 km de resolução horizontal e 38 níveis de resolução vertical. Nesse estudo, a assimilação dos dados de precipitação ocorre durante a geração do \textit{First Guess} pelo modelo Eta utilizando-se a técnica de \textit{Nudging}. Em seguinda, o sistema RPSAS gera uma análise onde são assimilados dados de observações provenientes do GTS. A partir desta análise, o modelo Eta produz previsões de 24 horas com saídas a cada 6 horas. Para a comparação das análises geradas pelo sistema RPSAS, foram utilizadas as análises de 40 km do CPTEC e reanálise do NCEO/DOE (versão 2). Para a comparação das previsões produzidas pelo modelo Eta, foram utilizados dados observacionais de precipitação dos projetos SALLJEX, TRMM e GPCP. Para a avaliação dos resultados das simulações numéricas, foram consideradas as seguintes estatísticas: Viés, Erro Quadrático Médio e Correlação de Anomalia. Foram avaliadas as seguintes variáveis: altura geopotencial, vento (componentes zonal e meridional), umidade específica e relativa, precipitação e temperatura a dois metros. Um estudo de caso foi realizado para verificar a acurácia do sistema de assimilação de precipitação e do modelo de previsão. Nos resultados encontrou-se que, no experimento que assimila precipitação, os valores de erro (Viés e Erro Quadrático Médio) em relação ao experimento sem assimilação de precipitação. Em comparação com as observações de precipitação, observou-se também que o experimento com assimilação de precipitação apresenta resultados mais satisfatórios do que o experimento sem assimilação de precipitação. De forma geral, observou-se também um incremento na qualidade das análises geradas pelo sistema RPSAS devido ao fato de que a precipitação foi assimilada durante a geração do \textit{first guess}. Além disso, no estudo de caso, o experimento com assimilação de precipitação foi capaz de detectar um caso de Complexo Convectivo de Mesoescala ocorrido em 23 de janeiro de 2003.

\end{resumo}
