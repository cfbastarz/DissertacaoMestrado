%% ABSTRACT
\begin{abstract}
\hypertarget{estilo:abstract}{}

In this work a data assimilation study was performed to assess the impact of assimilation of estimated precipitation from TRMM on the CPTEC's RPSAS analyses and the EtaWS model forecast over the region of La Plata Basin, during a case o MCC occurred between 22th and 23th January 2003. The RPSAS data assimilation system and the EtaWS mesoscale regional model (both with 20km of spatial resolution) were run together with and without the TRMM precipitation. Is this study the assimilation of precipitation is basically a nudging process and is performed during the first guess stage by the Eta model, like in the NCEP (National Centers for Environmental Predictions) EDAS (Eta Data Assimilation System) precipitation data assimilation. During this process the model adjusts the precipitation by comparing, at which grid point and at which time step, the model precipitation against the TRMM precipitation. Doing this some adjustments are made on the latent heat vertical profile, water vapor mixing ratio (qv) and relative humidity (RH), by considering the Betts-Miller-Janjic convective parameterization. On the next step, the RPSAS produces an analysis which covers most of the South America and the adjacent oceans. From this analysis the Eta model produces 6h, 12h, 18h and 24h forecast. Data collected from the SALLJEX (South America Low Level Jet EXperiment) was used to compare the forecasts of the model and the CPTEC 40km Regional Reanalysis was used to compare with the RPSAS analyses. Some preliminary results show that the precipitation assimilation improves the first hours of the forecast (typically 6h). The variables verified were the u and v wind fields, geopotential height field and the precipitation field. The convective precipitation field were improved, mainly over the 6h forecast. This is an important improvement because the first guess field will serve as an analysis of the next forecast window. Also were noticed that the mean error for those variables was reduced (principally for the u field). This reveals that with an improved first guess field, the model was able to detect the MCC occurred in the north of Argentina, due to the improved representation of the winds fields (direction and intensity), pressure and the surface variables.

\end{abstract}
