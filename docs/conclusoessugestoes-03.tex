%% CAPITULO 5
\hypertarget{estilo:capitulo}{}
\chapter{CONCLUSÕES E SUGESTÕES}
\label{ss:cap5}

A inclusão da precipitação no ciclo do sistema Eta+RPSAS, tem como principal objetivo reduzir o tempo de \textit{spin up} do modelo Eta e melhorar as condições iniciais de temperatura e umidade do solo, e consequentemente aprimorar a representação dos sistemas convectivos, especialmente aqueles relacionados à convecção profunda, tais como os SCMs. Nesta dissertação de mestrado foi investigado o impacto causado pela inclusão dos dados estimados de precipitação do TRMM no ciclo de análises e previsões do sistema Eta+RPSAS, através da avaliação estatística do \textit{skill} do modelo Eta e de um estudo de caso de CCM.

Em relação às análises produzidas pelo sistema RPSAS, em comparação com as reanálises do NCEP e do CPTEC, o Viés e o Erro Quadrático Médio, em geral, apresentaram valores menores para o experimento CAP (com assimilação de precipitação) e valores maiores para o experimento SAP (sem assimilação de precipitação). Nesta avaliação, os melhores resultados foram encontrados para
a Altura Geopotencial em 500 hPa e em 250 hPa, a Temperatura do Ar e Ventos Zonal e Meridional em 250 hPa e Umidade Relativa em 850 hPa.

A partir da avaliação do \textit{skill} do modelo (cálculo do Viés, Erro Quadrático Médio e Correlação de Anomalia - calculados em relação às reanálises do NCEP), pôde-se notar que com a assimilação de precipitação houve uma pequena redução do Viés e do Erro Quadrático Médio, significando uma redução sensível do erro sistemático do modelo Eta no prognóstico das variáveis verificadas (Altura Geopotencial, Ventos Zonal e Meridional e Temperatura). No entanto, não foram encontradas grandes melhorias em relação à acurácia (Correlação de Anomalia) da previsão do modelo Eta para a previsão de 24 horas. Parte deste resultado pode ser devido à metodologia (dados e método de avaliação) adotada para o trabalho. Possivelmente, a utilização das taxas instantâneas de precipitação do TRMM 3B41 sejam mais apropriadas para inclusão no ciclo do sistema Eta+RPSAS, devido à sua amostragem espaço-temporal. Sistemas de mesoescala tendem a deslocar-se e intensificar-se rapidamente em um curto período tempo (tipicamente 1 hora). Portanto, em 3 horas um sistema desse gênero pode evoluir rapidamente em seu ciclo de vida. A idéia principal da inclusão de precipitação é corrigir instantaneamente as condições iniciais de solo (e consequentemente atmosféricas) e que modulam o desenvolvimento convectivo desses sistemas melhorando o diagnóstico da precipitação pelo sistema de análise/previsão.

Durante os meses de Novembro de 2002 a Fevereiro de 2003, foi realizado um dos maiores experimentos de campo - SALLJEX, cujo principal objetivo foi estudar os JBN e sua influência na dinâmica dos processos convectivos de mesoescala. Durante o mês de Janeiro de 2003, foi verificada a ocorrência de mais de 100 SCMs, dos quais 2 foram classificados como CCM \cite{zipseretal04}. Dentre estes dois evento de precipitação severa foi escolhido o ocorrido no dia 23 de Janeiro de 2003 para a realização de um estudo de caso. Nas simulações com (CAP) e sem (SAP) assimilação de precipitação do TRMM, pôde-se concluir que o modelo regional de previsão de tempo Eta foi capaz de simular algumas das características do CCM. Vale ressaltar, que uma característica dos modelos regionais de mesoescala como o Eta, na previsão de sistemas convectivos, é atrasar a posição e subestimar a quantidade de precipitação associada e esse tipo de evento precipitante, principalmente quando se utiliza a aproximação hidrostática. Este fato pôde ser verificado com a simulação SAP do estudo de caso, em que o modelo não previu a precipitação associada ao CCM sobre o norte da Argentina. O experimento CAP, que assimilou a precipitação do TRMM, foi capaz de detectar a presença da precipitação associada ao CCM e de posicionar e simular uma quantidade de precipitação compatível com a observada, mas não muito significativa. Os dados de alta resolução do SALLJEX, no entanto, apresentam acumulados muito superiores (em torno de 160 mm contra 36 mm que foram diagnosticados pelo modelo Eta no experimento CAP para o dia 23 Janeiro de 2003) àquela apresentada pelo modelo de previsão e também pelos acumulados do TRMM. Esta situação se deve ao fato de que durante o período de estudos do SALLJEX foi utilizada uma rede de pluviômetros muito densa, o que operacionalmente, não é comum. Isto significa que os dados do SALLJEX apresentam melhor resolução e riqueza de detalhamento dos campos de precipitação.

Em suma, pode-se dizer que a Assimilação de Precipitação no sistema de assimilação de dados Eta+RPSAS apresentou resultados satisfatórios, principalmente para a previsão de curtíssimo prazo (6 horas), tendo sido observado algum ganho de desempenho do modelo durante a geração do \textit{first guess} (uma vez que com a assimilação de precipitação, os campos de \textit{first guess} já estão balanceados quando a análise é gerada, o que também implica na redução do \textit{spin up} - que não foi quantificado - do modelo). Os resultados mais sensíveis podem ser observados em baixos e em médios níveis do que em altos níveis.

A inclusão de precipitação o ciclo de assimilação de dados do sistema Eta+RPSAS apresenta-se como um produto com potencial que vem de encontro com os objetivos do CPTEC devido aos resultados encontrados. No entanto, há alguns fatores que devem ser considerados para uma possível operacionalização desse sistema:

\begin{itemize}
\item Dados de precipitação do TRMM não estão disponíveis em tempo real para assimilação;
\item Mesmo havendo disponibilidade destes dados de precipitação, há que se considerar um conjunto mais denso de dados de precipitação, como por exemplo, TRMM+Hidroestimador+Observações para garantir um mínimo de informações de precipitação para inicializar o modelo. Em comparação com o EDAS do NCEP, o sistema de observações utilizado para a assimilação de precipitação é muito denso;
\item Utilização das estimativas/observações de precipitação com resolução temporal compatível com o esquema de assimilação proposto.
\end{itemize}

Estes fatores devem ser considerados, pois como foi mostrado em alguns resultados, a assimilação de precipitação exerce uma influência sensível nas condições iniciais do solo (principalmente umidade), que são fundamentais para o desenvolvimento dos processos convectivos. Em condições mais adequadas (considerando-se o emprego adequado da metodologia um conjunto de dados de precipitação mais denso), este resultados tenderão a ser melhores.

De outro ponto de vista, a simplicidade de todo o processo de inclusão de precipitação, não causa nenhum ônus ao desempenho geral do sistema Eta+RPSAS. Além disso, no fututo, outros modelos (com diferentes aproximações físicas e dinâmicas) e sistemas mais sofisticados de AD (e. g., filtro de Kalman) também poderão utilziar este mesmo esquema de assimilação de precipitação.

Portanto, para estudos futuros que venham a utilizar este mesmo sistema de análises/previsões (Eta+RPSAS), sugere-se o seguinte:

\begin{enumerate}
\item Comparar da acurácia do modelo Eta com a parametrização convectiva de Kain-Fritsch \textit{versus} Betts-Miller-Janjić considerando-se a assimilação de precipitação com análise gerada com e sem o RPSAS;
\item Verificar o esquema de inclusão de precipitação com dados horários do TRMM, Hidroestimador e outros dados de precipitação disponíveis;
\item Realizar experimentos com previsões livres mais longas (e.g. 72h) e avaliar o \textit{skill} do modelo Eta em outros períodos;
\item Realizar experimentos com as mesmas configurações adotadas para os experimentos desta dissertação (SAP e CAP) com uma resolução maior e no modo não-hidrostático buscando explorar se o sistema é capaz de simular maiores detalhes dos processos convectivos. Neste caso sugere-se utilizar as configurações determinadas por \citeonline{rozantecavalcanti07}, mas com a inclusão de precipitação e também com e sem a análise gerada pelo sistema RPSAS.
\end{enumerate}
