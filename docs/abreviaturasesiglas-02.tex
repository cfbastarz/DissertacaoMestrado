%% ABREVIATURAS E SIGLAS
\begin{abreviaturasesiglas}
3DVAR &--& \textit{Three Dimensional Variational Data Assimilation} \\
4DVAR &--& \textit{Four Dimensional Variational Data Assimilation} \\
AIRS &--& \textit{Atmospheric InfraRed Sounder} \\
AD &--& Assimilação de Dados \\
AI &--& \textit{Analysis Increment} \\
AO &--& Análise Objetiva \\
AS &--& América do Sul \\
ATOVS &--& \textit{Advanced Tiros Operational Vertical Sounder} \\
BRAMS &--& \textit{Brazilian Regional Atmospheric Modeling System} \\
BMJ &--& Betts-Miller-Janjić \\
CA &--& Correlação de Anomalia \\
CAD &--& Ciclo de Assimilação de Dados \\
CAP &--& Com Assimilação de Precipitação \\
CC &--& Condição de Contorno \\
CCM &--& Complexo Convectivo de Mesoescala \\
CI &--& Condição Inicial \\
CIF &--& Condição Inicial Filtrada \\
CLIVAR &--& \textit{Climate Variability and Predictability} \\
CFD &--& Com Filtro Digital \\
COLA &--& \textit{Center for Ocean Land Atmosphere Studies} \\
COMET &--& \textit{Cooperative Program for Operational Meteorology, Education and Training} \\
CPTEC &--& Centro de Previsão de Tempo e Estudos Climáticos \\
DOE &--& \textit{Department Of Energy} \\
DSA &--& Divisão de Satélites e Sistemas Ambientais \\
ECMWF &--& \textit{European Centre for Medium Range Weather Forecasts} \\
EDAS &--& Eta \textit{Data Assimilation System} \\
EnKF &--& \textit{Ensemble Kalman Filter} \\
EQM &--& Erro Quadrático Médio \\
Eta &--& Modelo de previsão de tempo regional de coordenada Eta  \\
FD &--& Filtro Digital \\
GFS &--& \textit{Global Forecast System} \\
GDAS &--& \textit{Global Data Assimilation System} \\
GMAO &--& \textit{Global Modeling and Assimilation Office} \\
GOES &--& \textit{Geostationary Satellite Server} \\
GrADS &--& \textit{Gria Analysis and Display System} \\
GTS &--& \textit{Global Telecommunication System} \\
GPSAS &--& \textit{Global Physical-space Statistical Analysis System} \\
IF &--& Inicialização Física \\
IFD &--& Inicialiazação por Filtro Digital \\
IOP &--& \textit{Intensive Observation Period} \\
JAXA &--& \textit{Japan Aerospace eXploration Agency} \\
JBN &--& Jatos de Baixos Níveis \\
JP &--& Jato Polar \\
JST &--& Jato Subtropical \\
JPN &--& Jato Polar Norte \\
JPS &--& Jato Polar Sul \\
LEKF &--& \textit{Local Ensemble Kalman Filter} \\
LETKF &--& \textit{Local Ensemble Transform Kalman Filter} \\
LSM &--& \textit{Land Surface Model} \\
MNN &--& Modos Normais Não-Lineares \\
NASA &--& \textit{National Aeronautics and Space Administration} \\
NCEP &--& \textit{National Centers for Environmental Predictions} \\
NOAH &--& \textit{Ncep, Oregon state university, Air force, Hydrological research} \\
OMF &--& \textit{Observation Minus Forecast} \\
OSU &--& \textit{Oregon State University Land Surface Model} \\
PIBAL &--& \textit{Pilot Balloon} \\
PNT &--& Previsão Numérica de Tempo \\
PSAS &--& \textit{Physical-space Statistical Analysis System} \\
\textit{Quik}SCAT &--& \textit{Quick Scatterometer} \\
RAOB &--& \textit{Rawinsonde Observation} \\
RPSAS &--& \textit{Regional Physical-space Statistical Analysis System} \\
SALLJ &--& \textit{South American Low Level Jet} \\
SALLJEX &--& \textit{South American Low Level Jet EXperiment} \\
SAP &--& Sem Assimilação de Precipitação \\
SATOB &--& \textit{Satellite Temps and Radiance Balance} \\
SCM &--& Sistemas Convectivos de Mesoescala \\
SOP &--& \textit{Special Observation Period} \\
SYNOP &--& \textit{Surface Synoptic Observations} \\
TEMP &--& \textit{Upper air Temperature} \\
TRMM &--& \textit{Tropical Rainfall Measuring Mission} \\
USGS &--& \textit{United States Geological Survey} \\
VAMOS &--& \textit{Variability of the American Monsoon Systems} \\
ZCAS &--& Zona de Convergência do Atlântico Sul \\
ZCIT &--& Zona de Convergência Intertropical \\
\end{abreviaturasesiglas}
