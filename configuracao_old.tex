%% CAPA DO DOCUMENTO
\instituicao{MINISTÉRIO DA CIÊNCIA E TECNOLOGIA}
\instituicaol{INSTITUTO NACIONAL DE PESQUISAS ESPACIAIS}
\instituicaosigla{INPE}
\instituicaocidade{São José dos Campos}
\author{Carlos Frederico Bastarz}
\email{pubtc@sid.inpe.br}
\titulo{IMPACTO DA ASSIMILAÇÃO DE DADOS DE PRECIPITAÇÃO NO SISTEMA RPSAS/CPTEC: UM ESTUDO DE CASO DE COMPLEXOS CONVECTIVOS DE MESOESCALA}
\date{2010}
\serieinpe{INPE-13269-MAN/45 - versão 1}
\descriccao{Dissertação de Mestrado do Curso de Pós-Graduação em Meteorologia, orientada pelos Drs. Dirceu Luis Herdies e Julio Pablo Reyes Fernandez, aprovada em XX de Março de 2010.}
\repositorio{sid.inpe.br/iris@1916/2005/05.19.15.27}

%% VERSO DA CAPA
\tituloverso{\vspace{-0.9cm}\textbf{Publicado por:}}
\descriccaoverso{Instituto Nacional de Pesquisas Espaciais - INPE\\
Gabinete do Diretor (GB)\\
Serviço de Informação e Documentação (SID)\\
Caixa Postal 515 - CEP 12.245-970\\
São José dos Campos - SP - Brasil\\
Tel.:(012) 3945-6911/6923\\
Fax: (012) 3945-6919\\
E-mail: {\url{pubtc@sid.inpe.br}}
}
\descriccaoversoA{\textbf{CONSELHO DE EDITORAÇÃO:}\\
\textbf{Presidente:}\\
Dr. Gerald Jean Francis Banon - Coordenação Observação da Terra (OBT)\\ 
\textbf{Membros:}\\
Dra. Maria do Carmo de Andrade Nono - Conselho de Pós-Graduação \\
Dr. Haroldo Fraga de Campos Velho - Centro de Tecnologias Especiais (CTE)\\
Dra. Inez Staciarini Batista - Coordenação Ciências Espaciais e Atmosféricas (CEA)\\
Marciana Leite Ribeiro - Serviço de Informação e Documentação (SID)\\
Dr. Ralf Gielow - Centro de Previsão de Tempo  e Estudos Climáticos (CPT)\\
Dr. Wilson Yamaguti - Coordenação Engenharia e Tecnologia Espacial (ETE)\\
\textbf{BIBLIOTECA DIGITAL:}\\
Dr. Gerald Jean Francis Banon - Coordenação de Observação da Terra (OBT)\\
Marciana Leite Ribeiro - Serviço de Informação e Documentação (SID)\\ 
Jefferson Andrade Ancelmo - Serviço de Informação e Documentação (SID)\\
Simone A. Del-Ducca Barbedo - Serviço de Informação e Documentação (SID)\\
\textbf{REVISÃO E NORMALIZAÇÃO DOCUMENTÁRIA:}\\
Marciana Leite Ribeiro - Serviço de Informação e Documentação (SID) \\
Marilúcia Santos Melo Cid - Serviço de Informação e Documentação (SID)\\
Yolanda Ribeiro da Silva e Souza - Serviço de Informação e Documentação (SID)\\
\textbf{EDITORAÇÃO ELETRÔNICA:}\\
Viveca Sant'Ana Lemos - Serviço de Informação e Documentação (SID)\\
}

%% FOLHA DE ROSTO
\instituicao{MINISTÉRIO DA CIÊNCIA E TECNOLOGIA}
\instituicaol{INSTITUTO NACIONAL DE PESQUISAS ESPACIAIS}
\instituicaosigla{INPE}
\instituicaocidade{São José dos Campos}
\author{Carlos Frederico Bastarz}
\email{pubtc@sid.inpe.br}
\titulo{IMPACTO DA ASSIMILAÇÃO DE DADOS DE PRECIPITAÇÃO NO SISTEMA RPSAS/CPTEC: UM ESTUDO DE CASO DE COMPLEXOS CONVECTIVOS DE MESOESCALA}
\title{IMPACT OF PRECIPITATION ASSIMILATION IN CPTEC'S RPSAS SYSTEM: A CASE STUDY OF MESOSCALE CONVECTIVE SYSTEMS}
\date{2010}
\serieinpe{INPE-13269-MAN/45 - versão 1}
\descriccao{Dissertação de Mestrado do Curso de Pós-Graduação em Meteorologia, orientada pelos Drs. Dirceu Luis Herdies e Julio Pablo Reyes Fernandez, aprovada em XX de Março de 2010.} 

%% FICHA CATALOGRAFICA
\cutterFICHAC{Cutter}
\autorUltimoNomeFICHAC{Bastarz}
\autorAbreviadoFICHAC {C. F.}
\tituloFICHAC{IMPACTO DA ASSIMILAÇÃO DE DADOS DE PRECIPITAÇÃO NO SISTEMA RPSAS/CPTEC: UM ESTUDO DE CASO DE COMPLEXOS CONVECTIVOS DE MESOESCALA}
\paginasFICHAC{\pageref{LastPage}}
\palavraschaveFICHAC{1.~Assimilação de Dados. 2.~Precipitação. 3.~Inicialização Física. 4.~TRMM. 5.~Complexos Convectivos de Mesoescala. I.~\mbox{Título}.}
\numeroCDUFICHAC{}

% NOTA DA FICHA (PARA TD)
\tipoTD{Dissertação Mestrado em Meteorologia}
\instituicaoDefesa{Instituto Nacional de Pesquisas Espaciais}
\anoDefesa{2010}

%% FOLHA DE APROVACAO PELA BANCA EXAMINADORA
\tituloFA{\textbf{ATENÇÃO! A FOLHA DE APROVAÇÃO SERÁ INCLUIDA POSTERIORMENTE.}}
\cursoFA{\textbf{}}
\candidatoOUcandidataFA{}
\dataAprovacaoFA{}
\membroA{}{}{}
\membroB{}{}{}
\membroC{}{}{}
\membroD{}{}{}
\membroE{}{}{}
\membroF{}{}{}
\membroG{}{}{}
\ifpdf

%% NIVEL DE COMPRESSAO {0 -- 9}
\pdfcompresslevel 9
\fi

%% DEFINE EM 80% A LARGURA DAS FIGURAS
%\newlength{\mylenfig} 
%\setlength{\mylenfig}{0.8\textwidth}

%% COMANDOS PESSOAIS
\newcommand{\vetor}[1]{\mathit{\mathbf{#1}}}
